%% Example of a LaTeX source file for a COLING-2012 submission
%% last updated: July 10, 2012
%% Optional instructions for authors within the tex file are provided as comments and start with 'for authors:...'
\documentclass[10pt,a5paper,twoside]{article}
\usepackage{coling2012}
\usepackage{comment}
%\title{Translating to Shakespeare: A Case Study in Paraphrasing Writing Styles}
\title{You can be Shakespeare! \\ A Case Study in Paraphrase Targeting Writing Styles}
%for authors: in case of more than four author names ref. to commented line below 
%\author{$Annie~SMITH^{1, 2}~~~LI~Xiao Dong^{1, 3}$\\$~~~Third~Author^{1, 2}~~~Fourth~Author^{1, 3}~~~ Fifth~Author^{2, 3}$\\
\author{$Author1^{1, 2}~~~Author2^{1, 3}$\\
{\small  	(1) INSTITUTE\_1, address 1\\ 
 		(2) INSTITUTE\_2, address 2\\
		(3) INSTITUTE\_3, address 3\\
  \texttt{author1@institute1, author@institute2} \\ 
}}

\begin{document}
\maketitle
%% The first mandatory ABSTRACT (\abstractEn) section below is for the English language
\abstractEn{  %ABSTRACT}{
We present initial investigation into the task of paraphrasing language while targeting a particular writing style.
The plays of William Shakespeare and their modern translations are used as a testbed for evaluating
paraphrase systems targeting a specific style of writing.
%We demonstrate that existing evaluation metrics developed in the Machine Translation and Paraphrase communities are
%insufficient when the goal is to generate paraphrases targeting a specific style, and
%propose a series of new metrics to measure how closely the generated paraphrases match the target
%style.  
We empirically show that even with a relatively small amount of parallel training data available, it is
possible to learn paraphrase models which capture stylistic phenomenon, and these models outperform
baselines based on dictionaries and out-of-domain parallel text.
In addition we present an initial investigation into automatic evaluation metrics for paraphrasing writing style.
To the best of our knowledge this is the first work to investigate the task of
paraphrasing text with the goal of targeting a specific style of writing.
}

\keywordsEn{Paraphrase, Writing Style}

\section{Introduction}
%The plays of William Shakespeare and their \emph{modern} translations are treated
%as parallel text which is used to learn paraphrase models targeting the style of Early Modern English employed by Shakespeare.

Identical meaning can be expressed or \emph{paraphrased} in many different ways; automatically detecting or generating different expressions with the same meaning is 
fundamental to many natural language understanding tasks\cite{Giampiccolo07}, so much previous work has investigated methods for automatic paraphrasing\cite{Barzilay03,dolan04,Shinyama03,Das09,bannard05}.  
Although two utternaces may be semantically equivelant, they can still be stylistically quite different.  For example, the same information information about a new product
is likely to be conveyed using very different lexical and grammatical patterns in advertising materials v.s. technical manuals, or in Shakespearean plays v.s. Hollywood movies.

In this paper, we investigate the task of automatic paraphrasing when targeting a specific writing style, focusing specifically on the style of Early Modern English employed by William Shakespeare.
We exploit modern translations of 17 plays written to help literature students to better understand Shakespeare.  These modern translations are used to generate a parallel corpus of Shakespere's style and modern English,
which is then used to train phrase-based translation models which are capable of automatically paraphrasing ordinary sentences into Shakesperean English.  In addition we develop several
baseline systems which don't make use of this source of parallel text and instead rely on dictionaries of expressions commonly found in shakesperean english, or parallel monolingual text
gathered through Amazon's Mechanical Turk \cite{chen11}.

We evaluate these models both through human judgements and standard evaluation metrics from the Machine Translation and paraphrase literature, however no previous work has investigated the ability of these automatic metrics
to capture the notion of writing style.  We propose several new metrics for evaluating the task of paraphrasing while targeting a specific style, and measure correlation with human judgement showing
promising results for this particular style.

Systems which are capable of automatically paraphrasing literary writing styles could be directly benefical for educational applications, for example helping students to experiment with writing literature in the
style of authors they are studying.  Additionally note that out of the 37 surviving plays written by William Shakespeare, only 17 currently have modern translations available; although we have not yet formally evaluated
paraphrasing in the other direction, this work also has the potential to make the other 20 plays more accessable to students of Shakespeare.

\begin{comment}
\begin{itemize}
  \item Define what we mean by writing styles.
  \item Define the paraphrasing task and describe previous work.
  \item Motivate the need for paraphrasing targeting a specific writing style (e.g. students of literature in a specific style, or helping people to understand documents written in an esoteric style).
   \begin{itemize}
     \item Mention several domains where paraphrasing into/out of a specific writing style could be beneficial (e.g. technical manuals, legal documents, etc...)
   \end{itemize}
  \item Summarize the main contributions.
\end{itemize}
\end{comment}

\section{Data}
We propose to use Shakespeare's plays and their modern English translations as a testbed the task of paraphrasing targeting a specific writing style.  Having access to parallel text in the target
style allows us to train statistical models for generating paraphrases, and also perform automatic evaluation using BLEU which requires access to a set of reference translations.  For this purpose
we scraped modern translations of 17 Shakespeare plays from \url{http://nfs.sparknotes.com}, and an additional 8 translations of overlapping plays from \url{http://enotes.com}, giving us
two reference translations for 8 out of the 17 plays.

After tokenizing and lowercasing, the plays and their modern translations were aligned using Bob Moore's bilingual sentence \cite{Moore02} aligner producing about 21,079 alignments out of 31,718 sentences
in the Sparknotes data, and 10,365 aligned sentence pairs out of 13,640 sentences in the enotes data.  In addition, we note that the modern translations from each source are qualitatively quite
different.  The Sparknotes paraphrases tend to differ more significantly from the original texts, whereas the enotes data tends to stick closer to the original text, being much more conservative
in it's use of paraphrase, although it does include many useful paraphrses which do tend to be qualitatively different from those used in the Sparknotes data. %TODO: need examples of differences between the corpora?
To illustrate these differences empircally, we evaluated a very simple paraphrase system using the standard BLEU evaluation metric which has been shown to be a useful measure of semantic equivelance
in Parapharse \cite{chen11}.
These corpus statistics are summarized in table \ref{corpus_stats}.

\begin{table}
  \begin{center}
    \begin{tabular}{|l|r|r|r|}
      \hline
      corpus & initial size & aligned size & No-Change BLEU\\
      \hline
      \hline
      \url{http://nfs.sparknotes.com} & 31,718 & 21,079 & 23.67 \\
      \hline
      \url{http://enotes.com} & 13,640 & 10,365 & 49.60 \\
      \hline
    \end{tabular}
  \end{center}
  \caption{Parallel corpora generated form modern translations of Shakespeare's plays}
  \label{corpus_stats}
\end{table}

\begin{itemize}
  \item Motivate the need for a benchmark dataset for evaluating the writing style paraphrase task
  \item Present Shakespeare / Modern translation data as a situation where we have parallel data available (useful for building models \& evaluating automatic evaluation metrics)
\end{itemize}

\section{Evaluation Metrics}
\begin{itemize}
  \item Describe the need for automatic evaluation metrics.
  \item Describe previously used evaluation metrics for paraphrase.
  \item Highlight problems with previous metrics when targeting a specific writing style.
  \item Propose new metrics.
\end{itemize}

\section{Experiments}
\begin{itemize}
  \item Experimental setup.
  \item Present results from human evaluation comparing various systems.
  \item Analyze correlation between evaluation metrics and human judgments.
\end{itemize}

\section{Related Work}
\begin{itemize}
  \item Kevin Knight's work on poetry generation
  \item Any work on writing style (e.g. classification)?  Possibly cite work on author attribution...
  \item work on paraphrase evaluation metrics (David Chen, CCB, etc...)
\end{itemize}

\section{Conclusions}

\bibliographystyle{apa}

\bibliography{paper.bib}

%%================================================================
\end{document}
